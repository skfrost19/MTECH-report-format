%%%%%%%%%%%%%%%%%%%%%%% CHAPTER - 3 %%%%%%%%%%%%%%%%%%%%\\
\chapter{Methodology}
\label{C3} %%%%%%%%%%%%%%%%%%%%%%%%%%%%
\clearpage
% \section{Proposed Methodology}
% The proposed methodology for fine-tuning LLMs includes the following steps:
% \begin{enumerate}
%   \item \textbf{Data Collection}: Data collection from public, freely available medical research papers from PubMed Central® \cite{pmc}.
%   \item \textbf{Data preparation}: Pre-processing the collected data, involving removal of author names, references, tables, figures, etc. Made use of Nvidia NIM API, llama3-70b-instruct for question-answer pair generation. \cite{llama3}
%   \item \textbf{Fine-tuning}: Fine-tuning the model on question-answer pairs.
%   \item \textbf{Model Evaluation}: Evaluation of trained model on benchmarking datasets.
% \end{enumerate}


% \section{Technical Aspects}
% \subsection{Gemini\_PMC}
% \begin{enumerate}
%   \item Made using vertex ai studio by Google
%   \item It makes use of an undisclosed predefined Parameter Efficient Fine Tuning (PEFT) method to fine-tune the base gemini model.
%   \item Number of epochs used in training : 4
%   \item Learning rate multiplier in training : 1
%   \item Fine-tuned on about 8000 question-answer pairs in jsonl format
%   \item Evaluation was done on MMLU datasets \cite{hendryckstest2021}
% \end{enumerate}
% The training loss for Gemini\_PMC is shown below, x axis represents the number of steps and y axis represents the loss value.
% \begin{figure}[H]
%   \centering
% \includegraphics[width=1.0\textwidth]{trainlossgeminipmc.png}
% \caption{Training loss of Gemini\_PMC model}
% \end{figure}

% \subsection{GPT2\_PMC}
% \begin{enumerate}
% \item Made use of Hugging Face Transformers Trainer library \cite{huggingface} to make gpt2\_pmc
% \item It is based on standard fine-tuning, meaning all parameters are updated
% \item Number of epochs used in training : 13
% \item Learning\_rate: 5e-05
% \item Optimizer: Adam with betas=\(0.9,0.999\) and epsilon=1e-08
% \item Train\_batch\_size: 2
% \item Eval\_batch\_size: 8
% \item Total\_train\_batch\_size: 16
% \item Fine-tuned on about 8000 question-answer pairs in csv format
% \item Automated evaluation on medical datasets was done using Open Medical-LLM Leaderboard \cite{openllmleaderboard}
% \end{enumerate}

\section*{Step 1: Data Collection}
\begin{itemize}
    \item \textbf{Biomedical Documents:} Collect a corpus of biomedical research papers
    \item \textbf{Queries:} Use a dataset which contains query-document pairs and relevance judgments. Or, create queries based on the domain.
    \item \textbf{Initial Document Retrieval:} Use an initial retrieval system to get a set of candidate documents for re-ranking.
\end{itemize}

\section*{Step 2: Preprocessing and Tokenization}
\begin{itemize}
    \item For each document and query, tokenize using each tokenizer and measure the difference in token length, word split, and vocabulary coverage.
    \item Analyze how tokenization handles biomedical terms like disease names, gene names, or abbreviations.
\end{itemize}

\section*{Step 3: Embedding Generation}
\begin{itemize}
    \item Use pre-trained LLMs to generate embeddings for the documents and queries.
    \item For each tokenizer, generate embeddings for both the documents and queries.
    \item Compare embeddings of documents  and queries from different tokenizers using cosine similarity and computational efficiency
\end{itemize}

\section*{Step 4: Re-Ranking Process}
\begin{itemize}
    \item Start with retrieving an initial list of candidate documents.
    \item Re-rank the documents based on their similarity scores, using embeddings generated from each tokenizer.
\end{itemize}

\section*{Step 5: Fine-Tuning for Domain Specificity}
\begin{itemize}
    \item Fine-tune the LLM using a pairwise ranking loss function
    \item Where the model learns to rank more relevant documents higher
\end{itemize}

\section*{Step 6: Evaluation}
\begin{itemize}
  \item Compare the re-ranking results with the baseline traditional ranking methods.
  \item Evaluate how different tokenizers (general-purpose vs. domain-specific) affect the rankings.
\end{itemize}